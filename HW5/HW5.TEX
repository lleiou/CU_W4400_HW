\documentclass[10pt]{article}
\usepackage[margin=1in]{geometry}          
\usepackage{graphicx}
\usepackage{amsthm, amsmath, amssymb}
\usepackage{setspace}\onehalfspacing
\usepackage[loose,nice]{units} %replace "nice" by "ugly" for units in upright fractions
\setlength{\parindent}{0pt}
\title{W4400 HW5}
\author{Ao Liu | al3472}
\begin{document}
\maketitle


1.


(a)


According to Lagrange Method, finding the maximum of $H=-\sum_{i=1}^{4}P_ilog_2P_i$  under $1=\sum_{i=1}^{4}p_i$ is finding the maximum of 

$$-\sum_{i=1}^{4}P_ilog_2P_i-\lambda (\sum_{i=1}^{4}p_i-1)$$

By calculating the derivative of $P_i$, we get 


\begin{align}
\frac{\partial}{\partial P_i}\{-\sum_{i=1}^{4}P_ilog_2P_i-\lambda (\sum_{i=1}^{4}p_i-1)\}&=0\nonumber\\
-log_2P_i-P_i\frac{\partial log_2P_i}{\partial P_i}-\lambda&=0\nonumber\\
P_i=2^{-\frac{1}{ln2}-\lambda}\nonumber
\end{align}

Since $1=\sum_{i=1}^{4}p_i$, we can calculate the value of the $\lambda=2-\frac{1}{ln2}$
\\\\
Thus, $$P_i=\frac{1}{4}, i=1,2,3,4$$which means that the variables follow discrete uniform distribution.
\\
(b)


Let $P_1=1, P_2=P_3=P_4=0$, then 
\begin{align}
H&=-P_1log_2P_1\nonumber\\
&=-log_2{1}\nonumber\\
&=0\nonumber
\end{align}
Since in a discrete space, the entropy $H0\geq0$
Thus, the minimum of the entropy is reached.

To sum up, by reversing the role of the four numbers, we finally get the following distributions:

$$P_1=1, P_2=P_3=P_4=0$$
$$P_2=1, P_1=P_3=P_4=0$$
$$P_3=1, P_1=P_2=P_4=0$$
$$P_4=1, P_1=P_2=P_3=0$$


(c)

Similar to question (a), $$P_i=\frac{1}{2^{n}}, i=1,...n$$
so we get the entropy:
$$H=-2^{n}\frac{1}{2^{n}}log_2(\frac{1}{2^{n}})=n$$


(d)

According to the situation given in the question, we have a transition matrix:
$$\textbf{p}=
\begin{bmatrix}
0.5&0.5\\0.5&0.5
\end{bmatrix}
$$
The initial distribution of the variables is:
$$
P=
\begin{bmatrix}
0.5\\  0.5
\end{bmatrix}
$$




(e)


This chain has lower entropy than the chain above, because it is proved that the chain above has the highest entropy, so this chain must have lower chain than that.


(f)


According to the situation given in the question, we have a transition matrix:
$$\textbf{p}=
\begin{bmatrix}
0.6&0.3\\0.4&0.7
\end{bmatrix}
$$

According to the conclusion we get about equilibrium distribution, $P_{eq}$ should be the eigenvector of $P$ corresponding to eigenvalue equals 1, which, after calculation, is
$$
P_{eq}=
\begin{bmatrix}
\frac{3}{7}\\  \frac{4}{7}
\end{bmatrix}
$$


(g)


$$\lim_{n\to\infty}P(X_n=0)=P_{eq}(X=0)=\frac{3}{7}$$
$$\lim_{n\to\infty}P(X_n=1)=P_{eq}(X=1)=\frac{4}{7}$$



(h)

According to the chain rule for entropy, we have $$H(X_1,X_2,...,X_n)=\sum_{i=1}^{n} H(X_i\vert X_1,...,X_{i-1})$$Then, according to the property of Markov Chain, we have  $$H(X_1,X_2,...,X_n)=H(X_1)+\sum_{i=2}^{n} H(X_i\vert X_{i-1})$$What's more, we know the fact that $$X_1\sim P_{eq}$$so $$H(X_i\vert X_{i-1})=H(X_2\vert X_{1})$$that is to say$$H(X_1,X_2,...,X_n)=H(X_1)+(n-1)H(X_2\vert X_{1})$$

$$H(X_1)=-(3/7)log_2(3/7)-(4/7)log_2(4/7)=0.9852$$

\begin{align}
H(X_2\vert X_1)&=3/7H(X_2 \vert X_1=0)+4/7H(X_2 \vert X_1=1)\nonumber\\
&=-3/7[0.6log_2(0.6)+0.4log_2(0.4)]-4/7[0.3log(0.3)+0.7log_2(0.7)]\nonumber\\
&=0.9197\nonumber
\end{align}

to sum up,
$$H(X_1,X_2,...,X_n)=0.9852+(n-1)0.9197$$



2.

(a)

\begin{figure}[htbp]%位置选项,[htbp]是浮动的意思,可以不全加入
\centering
\includegraphics[width=12cm]{photo}
\caption{The graph of $p(x; \theta)$ when $\theta=1$}
\label{fig:anna}
\end{figure}


(b)
The likelihood of this toy dataset is the product of the $p.d.f$ values of the individual data points.


(c)
The  ratio  of  the  likelihood: 
$$\frac{(2*e^{-2*1})*(2*e^{-2*2})*(2*e^{-2*4})}{(1*e^{-1*1})*(1*e^{-1*2})*(1*e^{-1*4})}=0.4611996<1$$
Thus, a higher rate would reduce the value of the likelihood.


(d)


By using Bayes' Formula, we have
\begin{align}
q(\theta \vert x)&\propto q(\theta)p(x\vert \theta) \nonumber\\
&\propto \theta^{\alpha_0-1}\beta_0^{\alpha_0}e^{-\beta_0\theta}\theta^{n}e^{-\sum_{i=1}^{n}x_i\theta}\nonumber\\
&=\theta^{(\alpha_0-1+n)}e^{-\theta(\beta_0+\sum_{i=1}^{n}x_i)}\nonumber
\end{align}
which shows that the posterior distribution of the $\theta$ is
$$Gamma(\alpha_0+n-1,\beta_0+\sum_{i=1}^{n}x_i)$$


(e)


$$\Pi(\theta\vert x_{1:n})=\frac{\Pi_{i=1}^{n}p(x_i\vert \theta)}{\Pi p(x_1)p(x_2)...p(x_n)}q(\theta)$$
similarly, $$\Pi(\theta\vert x_{1:n-1})=\frac{\Pi_{i=1}^{n-1}p(x_i\vert \theta)}{\Pi p(x_1)p(x_2)...p(x_{n-1})}q(\theta)$$

so the posterior can be rewritten as:
$$\Pi(\theta\vert x_{1:n})=\frac{p(x_n\vert\theta)}{p(x_n)}\Pi(\theta\vert x_{1:n-1})$$

which means the posterior $\Pi(\theta\vert x_{1:n})$ under $n$ observations can be computed as the posterior given a single observation $x_n$ using the prior $\Pi(\theta\vert x_{1:n-1})$
 
 
 (f)
 
 
 According to the conclusion that we got from question (d), (e), 
 $p(\theta\vert x_n)$ has distribution: $$Gamma(\alpha_{n-1}+1,\beta_{n-1}+x_n)$$

 
 
 
 
\end{document}
